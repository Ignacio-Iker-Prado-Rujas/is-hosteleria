%Documento de la Gestion de la configuracion sooftware o GCS, en la teoria el tema 7


\documentclass[spanish,a4paper,11pt, twoside]{report}	% Idioma, tamaño del papel, tamaño letra, documento (book, report, article, letter)

%%% PAQUETES
\usepackage[spanish,activeacute]{babel}				
% Babel: Adapta cosas como la tipografia, la fecha, lo de Chapter al español, y activeacute para apóstrofes (') como abreviaciones de acentos: \'{a}
\usepackage[utf8]{inputenc}					% Codificacion UTF8 (para meter tildes normal: á --> \'{a} )
\usepackage{multicol}						% Escritura en varias columnas
\usepackage{graphics}						% Inclusión de imágenes
\usepackage{graphicx}						% Mas para imagenes
\usepackage{geometry}						% Distribucion de la pagina: margenes, encabezados, tamaño pagina...
\usepackage{fancyhdr}						% Paquete para añadir y modificar encabezados y pies de pagina
\usepackage{hyperref}						% Para hipervínculos, en el indice al menos, GRACIAS A DAVID
%\usepackage{lastpage}						% Ultima pagina para poner, por ejemplo, 3 de 15
%%% PAQUETES MATEMATICOS
\usepackage{amsmath}						% Conjunto de paquetes desarrollados por la Amercian Matematical Society
\usepackage{amssymb}						% Tipografía mathbb y otros símbolos tambien de la AMS
\usepackage{amsthm}						% Paquete AMS theorem, de la AMS
\usepackage{amsfonts}						% Paquete con símbolos y mas, de la AMS
%\usepackage{nicefrac}						% Fracciones bonitas, LO DEJO COMENTADO PORQUE A VECES DA PROBLEMAS AL COMPILAR


%%% DECLARACIONES (sobre la forma de la pagina, encabezado etc.)
\pagenumbering{roman}						
% Para numerar las paginas en numeros romanos hasta que empiece el texto (tambien alph, Alph, roman, Roman...)
\pagestyle{fancy}							% Utiliza el paquete fancyhdr para encabezados y pies de pagina
%\thispagestyle{empty}  						% Para poner UNA pagina sin encabezados ni numero, "plain" CON numero, "fancy" normal
%\lhead{\section}							% Encabezado a la izquierda
%\fancyhead[RO,LE]{\bfseries Encabezado} 		%Encabezado de las páginas impares a la derecha y de las pares a la izquierda
\fancyhead[LO,RE]{\bfseries Gestión de la configuración} %Encabezado de las páginas impares a la izquierda y de las pares a la derecha
%\rhead{\bfseries Casos de uso}				%Encabezado a la derecha
\cfoot{\thepage}							% Numero de pagina centrado en el pie
%\cfoot{\thepage\ de \pageref{LastPage}}		% Numero de pagina centrado en el pie asi: n de m
\renewcommand{\headrulewidth}{0.4pt}			% Linea debajo del encabezado
\renewcommand{\footrulewidth}{0.4pt}			% Linea encima del pie de pagina
\renewcommand*{\thesection}{\arabic{section}}	% Hace que no apareca el indice de capitulos y que comience en section, GRACIAS A RUBEN
\newcommand*{\PKT}{\hbox{P}\kern-2.5pt\lower3.5pt\hbox{\small{K}}\kern-2.8pt\hbox{T}\kern-2pt}	%PiKey Team en bonito


%%%%% CUERPO %%%%%
\begin{document}

\title{\textbf{\huge{Gestión de la \\ 
	configuración Software}} \\ \vspace{0.3cm}
	\Large{Ingeniería del Software} \\
	\includegraphics[scale=0.3]{ucm.pdf}}
\author{{\Large{PiKey Team-}} \PKT \ : \vspace{0.2cm} \\
	Jesús Aguirre Pemán \\
	 Enrique Ballesteros Horcajo \\
	 Jaime Dan Porras Rhee \\
	 Ignacio Iker Prado Rujas \\
	 Alejandro Villarín Prieto }
\date{\Today}
\maketitle

\newpage
\mbox{}
\thispagestyle{empty}						% Hoja en blanco, sin numeros ni nada
\newpage


\tableofcontents 							%INDICE hipervinvulado

\newpage
\mbox{}
\thispagestyle{empty}						% Hoja en blanco, sin numeros ni nada
\newpage

\pagenumbering{arabic}						% Pone el contador de paginas a 1 y ahora en numeros normales

%%% PARTE 1: INTRODUCCIÓN______________________________________________________________________________________________________
\part{Introducción}
	\section{Propósito}
	\section{Alcance}
	%\section{Definición de términos clave} NO SE PONE -> GLOSARIO
	\section{Referencias}

\newpage
\mbox{}
\thispagestyle{empty}						% Hoja en blanco, sin numeros ni nada
\newpage

\setcounter{section}{0}

%%% PARTE 2: GESTIÓN DE LA GCS__________________________________________________________________________________________________
\part{Gestión de la GCS }
	\section{Organización}
	La GCS se realizará durante el curso 2012-2013 en  el grupo Pikey Team y se aplicará al proyecto de ingeniería del software kike-hostelería. Se realizarán  los cambios que indique Gonzalo, sin discusión. Los que nos indiquen
	los grupos que realicen las revisiones técnicas formales, si son aceptados por los participantes de nuestro grupo en las reuniones. Y los cambios indicados por los miembros del propio grupo que sean aceptados por al menos tres miembros.
	\section{Responsabilidades GCS}
	Las responsabilidades en la gestión serán indicadas por el jefe del proyecto, Íker. Una vez conocido el cambio que hay que realizar, éste indicará a cada miembro del grupo su parte del trabajo. En la elaboración del documento,
	Alejandro estará al cargo de la introducción y  Kike a cargo de la gestión. Jesús e Íker se encargarán de las actividades,  y Jaime se encargará de la gestión de los recursos, bajo la guía del jefe del proyecto.
	\section{Políticas, directivas y procedimientos aplicables}
	Las principales restricciones vienen indicadas por Gonzalo: principalmente no se realizarán ni el control de la configuración, las auditorías, el control de la interfaz, el control de las subcontrataciones, la planificación, y por ahora 
	tampoco el mantenimiento.

\newpage
\mbox{}
\thispagestyle{empty}						% Hoja en blanco, sin numeros ni nada
\newpage

\setcounter{section}{0}

%%% PARTE 3: ACTIVIDADES DE LA GCS_____________________________________________________________________________________________
\part{Actividades de la GCS}
	\section{Identificación de la configuración}
		A lo largo de la realización del proyecto hemos identificado los siguientes Elementos de Configuración Software:
		\subsection{Identificación de ECSs}
			% Lo que va en subsubsection => El NOMBRE del objeto es una cadena de caracteres que identifican al objeto sin ambigüedad
			% DESCRIPCION => El tipo de ECS (simple vs. compuesta), un identificador de proyecto, la información de la versión y/o cambio
			% RECURSOS => Entidades que proporciona, procesa, referencia o son, de alguna otra forma requeridas por el objeto
			% REALIZACION => Una referencia a la unidad de texto para objetos básicos y nulo para objetos compuestos
			
			% Jesús, creo que están todos los ECSs, si se te ocurren más añadelos, pero mi duda es a lo mejor quitar los de "introduccion de srs", "descripción de srs" y "introduccion de plan de proyecto" porque son partes un poco chorras, no se... 
			% A parte, he marcado todos los compuestos creo

			\subsubsection{\texttt{Documento de casos de uso}}
			\begin{itemize}	
				\item{Descripción:}
				\item{Lista de recursos:}
				\item{Lista de realización:}
			\end{itemize}	
	
			\subsubsection{Introducción del \texttt{Documento de especificación de requisitos Software}}
			\begin{itemize}	
				\item{Descripción:}
				\item{Lista de recursos:}
				\item{Lista de realización:}
			\end{itemize}		

			\subsubsection{Descripción general del \texttt{Documento de especificación de requisitos Software}}
			\begin{itemize}	
				\item{Descripción:}
				\item{Lista de recursos:}
				\item{Lista de realización:}
			\end{itemize}		

			\subsubsection{Requisitos específicos del \texttt{Documento de especificación de requisitos Software}}
			\begin{itemize}	
				\item{Descripción:}
				\item{Lista de recursos:}
				\item{Lista de realización:}
			\end{itemize}		

			\subsubsection{\texttt{Documento de especificación de requisitos Software}}
			\begin{itemize}	
				\item{Descripción:} Compuesto de...
				\item{Lista de recursos:}
				\item{Lista de realización:}
			\end{itemize}	

			\subsubsection{Introducción del documento \texttt{Plan de Proyecto Software}}
			\begin{itemize}	
				\item{Descripción:}
				\item{Lista de recursos:}
				\item{Lista de realización:}
			\end{itemize}		

			\subsubsection{Documento de \texttt{Estimación del proyecto Software}}
			\begin{itemize}	
				\item{Descripción:}
				\item{Lista de recursos:}
				\item{Lista de realización:}
			\end{itemize}		
			
			\subsubsection{Documento de \texttt{Gestión de riesgos Software}}
			\begin{itemize}	
				\item{Descripción:}
				\item{Lista de recursos:}
				\item{Lista de realización:}
			\end{itemize}		

			\subsubsection{Planificación temporal y recursos del proyecto del documento \texttt{Plan de Proyecto Software}}
			\begin{itemize}	
				\item{Descripción:}
				\item{Lista de recursos:}
				\item{Lista de realización:}
			\end{itemize}	

			\subsubsection{\texttt{Plan de Proyecto Software}}
			\begin{itemize}	
				\item{Descripción:} Compuesto de...
				\item{Lista de recursos:}
				\item{Lista de realización:}
			\end{itemize}	

			\subsubsection{Documento \texttt{Definiciones, acrónimos y abreviaturas}}
			\begin{itemize}	
				\item{Descripción:}
				\item{Lista de recursos:}
				\item{Lista de realización:}
			\end{itemize}		

			\subsubsection{Documento \texttt{Gestión de la configuración Software}}
			\begin{itemize}	
				\item{Descripción:}
				\item{Lista de recursos:}
				\item{Lista de realización:}
			\end{itemize}	

			\subsubsection{Documento \texttt{Garantía de calidad Software}}
			\begin{itemize}	
				\item{Descripción:} Compuesto de GCS... (creo que nada mas)
				\item{Lista de recursos:}
				\item{Lista de realización:}
			\end{itemize}	

			\subsubsection{Primera entrega}
			\begin{itemize}	
				\item{Descripción:} Compuesto de...
				\item{Lista de recursos:}
				\item{Lista de realización:}
			\end{itemize}	

			\subsubsection{Segunda entrega}
			\begin{itemize}	
				\item{Descripción:} Compuesto de...
				\item{Lista de recursos:}
				\item{Lista de realización:}
			\end{itemize}	

			\subsubsection{Tercera entrega}
			\begin{itemize}	
				\item{Descripción:}Compuesto de...
				\item{Lista de recursos:}
				\item{Lista de realización:}
			\end{itemize}	

			Por el momento no hay más ECSs, pero irán apareciendo más según se avance en el proyecto, como código, manuales de usuario, casos de prueba...

		\subsection{Nombrado de ECSs} % ¿¿??
		
		
		\subsection{Adquisición de ECSs} % ¿¿??


	\section{Contabilidad de estado de configuración} % ¿Alguien tiene idea de que coño es esto?


\newpage
\mbox{}
\thispagestyle{empty}						% Hoja en blanco, sin numeros ni nada
\newpage

\setcounter{section}{0}

%%% PARTE 5: RECURSOS DE LA GCS_______________________________________________________________________________________________
\part{Recursos de la GCS}

\newpage
\mbox{}
\thispagestyle{empty}						% Hoja en blanco, sin numeros ni nada al final del documento
\newpage

\end{document}

%ALEX
1. Introducción
	1.1 Propósito
	1.2 Alcance
	1.3 Definición de términos clave 
	1.4 Referencias

%KIKE
2. Gestión de la GCS 
	2.1 Organización
	2.2 Responsabilidades GCS
	2.3 Políticas, directivas y procedimientos aplicables

%JESÚS & IKER  (3.1 Y 3.3)
3. Actividades de la GCS
	3.1 Identificación de la configuración
		3.1.1 Identificación de ECSs 
		3.1.2 Nombrado de ECSs 
		3.1.3 Adquisición de ECSs
	3.2 Control de la configuración 					%ESTE NO HAY QUE HACERLO
		3.2.1 Petición de cambios
		3.2.2 Evaluación de cambios
		3.2.3 Aprobación o desaprobación de cambios 
		3.2.4 Implementación de cambios
	3.3 Contabilidad de estado de configuración 
	3.4 Auditorias y revisiones de la configuración		%ESTE NO HAY QUE HACERLO 
	3.5 Control de interfaz							%ESTE NO HAY QUE HACERLO
	3.6 Control de la subcontratación/compra			%ESTE NO HAY QUE HACERLO


4. Planificaciones de la GCS							%ESTE NO HAY QUE HACERLO

%JAIME
5. Recursos de la GCS


6. Mantenimiento del plan de GCS						%ESTE NO HAY QUE HACERLO
