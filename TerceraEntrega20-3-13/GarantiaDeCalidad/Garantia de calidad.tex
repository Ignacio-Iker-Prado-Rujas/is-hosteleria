%Documento de garantia de calidad software, en la teoría el tema 8


\documentclass[spanish,a4paper,11pt, twoside]{report}	% Idioma, tamaño del papel, tamaño letra, documento (book, report, article, letter)

%%% PAQUETES
\usepackage[spanish,activeacute]{babel}				
% Babel: Adapta cosas como la tipografia, la fecha, lo de Chapter al español, y activeacute para apóstrofes (') como abreviaciones de acentos: \'{a}
\usepackage[utf8]{inputenc}					% Codificacion UTF8 (para meter tildes normal: á --> \'{a} )
\usepackage{multicol}						% Escritura en varias columnas
\usepackage{graphics}						% Inclusión de imágenes
\usepackage{graphicx}						% Mas para imagenes
\usepackage{geometry}						% Distribucion de la pagina: margenes, encabezados, tamaño pagina...
\usepackage{fancyhdr}						% Paquete para añadir y modificar encabezados y pies de pagina
\usepackage{hyperref}						% Para hipervínculos, en el indice al menos, GRACIAS A DAVID
%\usepackage{lastpage}						% Ultima pagina para poner, por ejemplo, 3 de 15
%%% PAQUETES MATEMATICOS
\usepackage{amsmath}						% Conjunto de paquetes desarrollados por la Amercian Matematical Society
\usepackage{amssymb}						% Tipografía mathbb y otros símbolos tambien de la AMS
\usepackage{amsthm}						% Paquete AMS theorem, de la AMS
\usepackage{amsfonts}						% Paquete con símbolos y mas, de la AMS
%\usepackage{nicefrac}						% Fracciones bonitas, LO DEJO COMENTADO PORQUE A VECES DA PROBLEMAS AL COMPILAR


%%% DECLARACIONES (sobre la forma de la pagina, encabezado etc.)
\pagenumbering{roman}						
% Para numerar las paginas en numeros romanos hasta que empiece el texto (tambien alph, Alph, roman, Roman...)
\pagestyle{fancy}							% Utiliza el paquete fancyhdr para encabezados y pies de pagina
%\thispagestyle{empty}  						% Para poner UNA pagina sin encabezados ni numero, "plain" CON numero, "fancy" normal
%\lhead{\section}							% Encabezado a la izquierda
%\fancyhead[RO,LE]{\bfseries Encabezado} 		%Encabezado de las páginas impares a la derecha y de las pares a la izquierda
\fancyhead[LO,RE]{\bfseries Garantía de calidad} 	%Encabezado de las páginas impares a la izquierda y de las pares a la derecha
%\rhead{\bfseries Casos de uso}				%Encabezado a la derecha
\cfoot{\thepage}							% Numero de pagina centrado en el pie
%\cfoot{\thepage\ de \pageref{LastPage}}		% Numero de pagina centrado en el pie asi: n de m
\renewcommand{\headrulewidth}{0.4pt}			% Linea debajo del encabezado
\renewcommand{\footrulewidth}{0.4pt}			% Linea encima del pie de pagina
\renewcommand*{\thesection}{\arabic{section}}	% Hace que no apareca el indice de capitulos y que comience en section, GRACIAS A RUBEN
\newcommand*{\PKT}{\hbox{P}\kern-2.5pt\lower3.5pt\hbox{\small{K}}\kern-2.8pt\hbox{T}\kern-2pt}	%PiKey Team en bonito


%%%%% CUERPO %%%%%
\begin{document}

\renewcommand{\chaptername}{Parte}			% Renombrar "Capítulo" como "Parte"
\renewcommand{\thechapter}{\Roman{chapter}}	% Cambio la numeración de los capítulos a números romanos en mayúsculas

\title{\textbf{\huge{Garantía de \\ 
	calidad Software}} \\ \vspace{0.3cm}
	\Large{Ingeniería del Software} \\
	\includegraphics[scale=0.3]{ucm.pdf}}
\author{{\Large{PiKey Team-}} \PKT \ : \vspace{0.2cm} \\
	Jesús Aguirre Pemán \\
	 Enrique Ballesteros Horcajo \\
	 Jaime Dan Porras Rhee \\
	 Ignacio Iker Prado Rujas \\
	 Alejandro Villarín Prieto }
\date{\Today}
\maketitle

\newpage
\mbox{}
\thispagestyle{empty}						% Hoja en blanco, sin numeros ni nada
\newpage


\tableofcontents 							%INDICE hipervinvulado

\newpage
\mbox{}
\thispagestyle{empty}						% Hoja en blanco, sin numeros ni nada
\newpage

\pagenumbering{arabic}						% Pone el contador de paginas a 1 y ahora en numeros normales


%INTRODUCCIÓN 
\chapter{ Propósito del plan de garantía de calidad}

	
	 La Garantía de Calidad del Software consiste en un medio de seguimiento de los procesos de ingeniería de software y métodos utilizados para asegurar la calidad del software que se produce en el proyecto.\\
	
	La SQA abarca todo el proceso de desarrollo de software, incluyendo procesos tales como la definición de requerimientos, diseño de software, programación, control de código fuente, revisiones de código, gestión de cambios, gestión de configuración, pruebas o gestión de versiones.\\
	
	El plan de Garantía de Calidad del Software define las actividades específicas a llevar a cabo en un este proyecto. Contiene una lista de comprobación para las actividades que se deben llevar a cabo para asegurar la calidad del producto y garantizar que KIKE Hostelería® cumple los requisitos especificados en la documentación.\\
	%No hagais caso a estos comentarios, son de panchitos del internes
	%Para cada fase del proyecto, se debe crear un plan para su monitoreo.
	%Este documento pretende entregar la pauta general del proceso que debe seguir una Gerencia de SQA en una fábrica de software.
	La SQA engloba:

\begin{itemize}
	\item Enfoque de gestión de calidad.
	\item Tecnologías de IS (métodos y herramientas).
	\item Revisiones Técnicas Formales.
	\item Estrategia de pruebas.
	\item Control de la documentación y de cambios.
	\item Procedimientos que aseguren ajustes a los estándares de IS
	\item Mecanismos de medición y generación de informes
\end{itemize}	

	(-Delinea el propósito específico y el alcance del plan	SQA.

	- Lista los nombres de los elementos software cubiertos por el plan SQA y el uso de dichos elementos.
		
	- Determina la porción del ciclo de vida cubierta por el plan para cada elemento software.) Alguien sabe que carajo es esto?

\newpage
\mbox{}
\thispagestyle{empty}						% Hoja en blanco, sin numeros ni nada
\newpage

\chapter{ Documentos de referencia}
	Co, esto se hace al final cuando este acabado el documento
	(- Proporciona una lista completa de cualquier documento referenciado en el plan o utilizado en su elaboración.)

\newpage
\mbox{}
\thispagestyle{empty}						% Hoja en blanco, sin numeros ni nada
\newpage

\chapter{ Gestión}% by Kike
	La estructura en la gestión de la calidad en nuestro grupo es lineal, es decir, prácticamente todos los integrantes revisan cada documento. Esto siempre bajo la organización del 
	jefe del proyecto que indica a cada integrante qué parte de cada documento revisa. En cuanto al documento de sqa. Jesús se encarga de la redacción, Íker de la organización de las tareas, Kike de la 
	gestión, Alejandro de las herramientas y Jaime del mantenimiento.
	En la redacción se incluye la introducción del documento de sqa, la relación de los documentos de referencia, y la colección de registros, mantenimiento y conservación. En la organización se incluye la asignación de tareas, 
	la creación de los modelos para latex y el seguimiento del trabajo. En la gestión se registra la organización del equipo y de las tareas. En las herramientas se escribe la relación de tareas y en el mantenimiento se garantiza
	que el plan de garantía de calidad esté actualizado.


	(Tareas relacionadas con la SQA.
	- Idealmente redactado en formato IEEE Std. 1058-
	1998, IEEE Standard for Software Project
	Management Plans
	- Describe la estructura organizativa que influye y
	controla la calidad del software.
	- Identifica roles y responsabilidades dentro del plan
	SQA.
	- Identifica a los responsables de preparar y mantener
	el plan SQA.
	- Identifica las tareas asociadas al proceso SQA.
	- Identifica la relación entre estas tareas y las de la
	planificación temporal)

\newpage
\mbox{}
\thispagestyle{empty}						% Hoja en blanco, sin numeros ni nada
\newpage

\chapter{ Documentación}%By Jaime
	\section{Propósito}
	\section{Requisitos mínimos de documentación}
	\section{Otra documentación}

	(- Define toda la documentación que se va a generar
	durante el proceso de desarrollo.
	- Lista los documentos que serán revisados o
	auditados, así como los criterios de revisión.)

\newpage
\mbox{}
\thispagestyle{empty}						% Hoja en blanco, sin numeros ni nada
\newpage

\chapter{ Estándares, prácticas convenciones y métricas}
	(- Esta sección es un poco miscelánea en SQA.) %Osea, ¿¿la quitamos?? (by Iker). Es lo que dice el standard. Yo pondría algo de petaqueo

\newpage
\mbox{}
\thispagestyle{empty}						% Hoja en blanco, sin numeros ni nada
\newpage

\chapter{ Revisiones del software}
	\section{Propósito}
	La intención de las revisiones del software es detectar lo antes posible los fallos que exitan en el software que estamos desarrollando, con el fin de ahorrar costes en la corrección de estos fallos. Nos centraremos en las revisiones técnicas formales,
	y se las haremos a los documentos que hemos producido hasta ahora, es decir, al plan de proyecto, al documento de casos de uso, a la especificación de requisitos, y al documento de gestión de riesgos. Además de detectar posibles errores,
 	las revisiones nos permitirán asegurarnos de que nuestros documentos se ajustan a los estándares y de que se cumplen los requisitos especificados.

	\section{Requisitos mínimos}
	Mediante las revisiones nos aseguraremos de que no haya faltas de ortografía ni errores de expresión en ningún documento, y de que se respetan los estándares elegidos para cada documento.
	\section{Otras revisiones y auditorias}
	Por ahora, además de las revisiones técnicas formales, realizaremos revisiones internas llevadas a cabo por los mismos integrantes del  equipo de desarrollo;  y Gonzalo por su parte realizará varias revisiones independientes.

\newpage
\mbox{}
\thispagestyle{empty}						% Hoja en blanco, sin numeros ni nada
\newpage

\chapter{ Prueba}%By Jaime
	(- Identifica todas las pruebas no incluidas en el plan
	de verificación y validación.)

\newpage
\mbox{}
\thispagestyle{empty}						% Hoja en blanco, sin numeros ni nada
\newpage

\chapter{ \hspace{0.25cm}Informe de problemas y acción correctiva}
	(- Describe las prácticas y procedimientos de informe,
	seguimiento y resolución de problemas, tanto a nivel	
	producto como proceso.
	- Determina las responsabilidades organizativas
	relativas a su implementación.)

\newpage
\mbox{}
\thispagestyle{empty}						% Hoja en blanco, sin numeros ni nada
\newpage

\chapter{ Herramientas, técnicas y metodologías}
	(- Herramientas, técnicas y metodologías utilizadas
	para soportar el proceso de SQA.)

\newpage
\mbox{}
\thispagestyle{empty}						% Hoja en blanco, sin numeros ni nada
\newpage

\chapter{ Control de medios}
	(- Determina los métodos para:
	- Identificar el medio físico de cada producto software.
	- Protegerlo de daños durante el proceso.)

\newpage
\mbox{}
\thispagestyle{empty}						% Hoja en blanco, sin numeros ni nada
\newpage

\chapter{ Control de proveedor}
	(- Determina las técnicas para garantizar que el
	software proporcionado por proveedores externos
	cumple sus requisitos.
	- También es aplicable a código heredado.)

\newpage
\mbox{}
\thispagestyle{empty}						% Hoja en blanco, sin numeros ni nada
\newpage

\chapter{ Colección de registros, mantenimiento y conservación}
	(- Identifica la documentación SQA que no se debe
	tirar tras acabar el proceso.
	- Determina los métodos y medios para ensamblar,
	archivar, salvaguardar y mantener la documentación.
	- Fija el periodo de conservación de la información.)

\newpage
\mbox{}
\thispagestyle{empty}						% Hoja en blanco, sin numeros ni nada
\newpage

\chapter{ \hspace{0.25cm}Formación}
	La formación que han recibido los integrantes del grupo de desarrollo para satisfacer las necesidades del plan sqa, es la de la asignatura ingeniería del software de la ucm, impartida por Gonzalo Mendes. En ella se les ha enseñado 
	cómo hacer un desarrollo de software que se ajuste a los estándares más utilizados,  y cómo desarrollar un software de calidad que se ajuste a los requisitos y que satisfaga al cliente.
	(- Identifica las actividades de formación necesarias
	para satisfacer las necesidades del plan SQA.)

\newpage
\mbox{}
\thispagestyle{empty}						% Hoja en blanco, sin numeros ni nada
\newpage

\chapter{ \hspace{0.25cm}Gestión del riesgo}
	Está hecho en un documento aparte.

%\chapter{Glosario} (- Términos específicos del plan SQA.) 					EL GLOSARIO VA A PARTE

\newpage
\mbox{}
\thispagestyle{empty}						% Hoja en blanco, sin numeros ni nada
\newpage

\chapter{ Procedimiento de cambio e historia del plan SQA}
	(- Procedimientos de modificación del plan SQA.
	- Procedimientos de mantenimiento del historial de
	cambios.
	- Historial de cambios.)
	


\newpage
\mbox{}
\thispagestyle{empty}						% Hoja en blanco, sin numeros ni nada al final del documento
\newpage

\end{document}

%JESÚS, ya está empezado by Kike(no hacer mucho caso a lo que haya puesto kike)
1. Propósito

%JESÚS (creo que este es corto)
2. Documentos de referencia 

%KIKE. La concha, este nombre me perseguirá el resto de mi historia
3. Gestión

%JAIME (creo que es corto)
4. Documentación
	4.1 Propósito
	4.2 Requisitos mínimos de documentación 
	4.3 Otra documentación

%A LO MEJOR SE QUITA, COMO DIGA KIKE
5. Estándares, prácticas, convenciones y métricas 
	5.1 Propósito
	5.2 Contenido

%KIKE & IKER
6. Revisiones del software 
	6.1 Propósito
	6.2 Requisitos mínimos
	6.3 Otras revisiones y auditorias 

%JAIME (creo que este es corto)
7. Pruebas

%ALEX
8. Informe de errores y acciones correctoras 

%ALEX (creo que este es corto)
9. Herramientas, técnicas y metodologías 

%ALEX (creo que este es corto)
10. Control de medios

%JESUS (creo que este es corto)
11. Control de proveedor

%JESUS
12. Colección de registros, mantenimiento y conservación 

%IKER (npi de qué va esto, Kike ayudame por favor!!) ¿Qué concha son las actividades de formación? Jajajajajaj 
% Aquí hay que meter las clase de is que nos han dado. Es tanto la formación que se da a los que van a usar el software como la que ha recibido el equipo de is. Prometo que esto se lo pregunté en clase
% Me lo creo, con eso me vale. En resumen => PeTaKeo
13. Formación

%ESTA HECHO A PARTE BY ALEX & JAIME
14. Gestión del riesgo

%ESTA EN UN DOCUMENTO A PARTE
15. Glosario

%IKER
16. Procedimiento de cambio e historial del plan de SQA
